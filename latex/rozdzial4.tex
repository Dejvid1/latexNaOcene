\chapter{Co Dalej? Nauka w Praktyce, Projekty i Zasoby}

Zrozumienie teorii to pierwszy krok. Drugim jest praktyka.

\section{Przykładowe Projekty do Nauki OOPs}
\begin{itemize}
    \item \textbf{Projekt 1: Prosty System Biblioteczny.} Ćwiczy podstawy klas i hermetyzację.
    \item \textbf{Projekt 2: Gra Tekstowa (RPG).} Zastosowanie dziedziczenia i polimorfizmu.
    \item \textbf{Projekt 3: Symulator Bankomatu.} Ćwiczy kompozycję i obsługę wyjątków.
\end{itemize}

\section{Klasyczne Książki o OOPs}
Warto sięgnąć po literaturę uznawaną za kanon:

\begin{itemize}
    \item \textbf{,,Wzorce Projektowe''} (Banda Czterech) \cite{gof}.
    \item \textbf{,,Czysty kod''} (Robert C. Martin) \cite{cleancode}.
    \begin{figure}[h]
    \centering
    \includegraphics[width=0.4\textwidth]{image1.png}
    \caption{Okładka książki ,,Czysty kod'' Roberta C. Martina.}
    \label{fig:cleancode}
\end{figure}
    \item \textbf{,,Rusz głową! Wzorce projektowe''} \cite{headfirst}.
    \item \textbf{,,Refaktoryzacja''} (Martin Fowler) \cite{refactoring}.
\end{itemize}

\section{Inne Zasoby}
Warto również korzystać z platform e-learningowych, oficjalnej dokumentacji oraz analizować kod open source na GitHubie.