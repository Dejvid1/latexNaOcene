\chapter{Zaawansowane Koncepcje i Dobre Praktyki}

Efektywne programowanie wymaga znajomości wzorców i zasad, które pomagają tworzyć kod elastyczny.

\section{Konstruktory i Destruktory}
\begin{itemize}
    \item \textbf{Konstruktor:} Metoda wywoływana automatycznie przy tworzeniu obiektu (inicjalizacja).
    \item \textbf{Destruktor:} Metoda wywoływana przed zniszczeniem obiektu (zwalnianie zasobów).
\end{itemize}

\section{Dziedziczenie a Kompozycja}
Często lepszym podejściem od dziedziczenia (relacja ,,jest'') jest \textbf{kompozycja} (relacja ,,ma'').
\begin{itemize}
    \item Dziedziczenie: \texttt{Samochod} \textbf{jest} \texttt{Pojazdem}.
    \item Kompozycja: \texttt{Samochod} \textbf{ma} \texttt{Silnik}.
\end{itemize}
Kompozycja jest bardziej elastyczna i pozwala na luźniejsze powiązania między klasami.

\subsection{Porównanie: Dziedziczenie vs Kompozycja}
Poniższa tabela podsumowuje kluczowe różnice:

\begin{table}[h]
\centering
\renewcommand{\arraystretch}{1.5}
\begin{tabularx}{\textwidth}{|l|X|X|}
\hline
\textbf{Cecha} & \textbf{Dziedziczenie (,,IS-A'')} & \textbf{Kompozycja (,,HAS-A'')} \\ \hline
\textbf{Definicja} & Klasa pochodna dziedziczy po bazowej. & Klasa zawiera instancję innej klasy. \\ \hline
\textbf{Wiązanie} & Silne (\textit{tight coupling}). & Luźne (\textit{loose coupling}). \\ \hline
\textbf{Elastyczność} & Mniejsza (ustalana przy kompilacji). & Większa (możliwa wymiana w trakcie działania). \\ \hline
\end{tabularx}
\caption{Porównanie dziedziczenia i kompozycji.}
\end{table}

\section{Zasady SOLID}
Zbiór pięciu zasad SOLID pomaga tworzyć zrozumiałe oprogramowanie:
\begin{enumerate}
    \item \textbf{S (Single Responsibility):} Jedna klasa, jedna odpowiedzialność.
    \item \textbf{O (Open/Closed):} Otwarte na rozszerzenia, zamknięte na modyfikacje.
    \item \textbf{L (Liskov Substitution):} Podklasy muszą móc zastąpić klasy bazowe.
    \item \textbf{I (Interface Segregation):} Wiele małych interfejsów zamiast jednego dużego.
    \item \textbf{D (Dependency Inversion):} Zależność od abstrakcji, nie od konkretów.
\end{enumerate}