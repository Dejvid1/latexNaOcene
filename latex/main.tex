\documentclass[12pt, a4paper]{report}

\usepackage[utf8]{inputenc}
\usepackage[T1]{fontenc}
\usepackage{polski}
\usepackage{geometry}
\usepackage{fancyhdr}
\usepackage{titlesec}
\usepackage{tabularx}
\usepackage{hyperref}
\usepackage{csquotes}
\usepackage{graphicx}

\geometry{left=25mm, right=25mm, top=25mm, bottom=25mm}

\pagestyle{fancy}
\pagestyle{fancy}
\fancyhf{}
\fancyhead[L]{\nouppercase{\leftmark}}
\fancyhead[R]{Dawid Szuber}
\fancyfoot[C]{\thepage}
\fancypagestyle{plain}{
    \fancyhf{} 
    \fancyhead[L]{\nouppercase{\leftmark}}
    \fancyhead[R]{Dawid Szuber}
    \fancyfoot[C]{\thepage}
}
\title{\textbf{Programowanie Obiektowe:\\Teoria i Praktyka}}
\author{Dawid Szuber}
\date{03.12.2025}

\begin{document}

\maketitle

\begin{abstract}
\noindent
Niniejsza praca stanowi kompleksowe wprowadzenie do paradygmatu programowania obiektowego (OOP). Opracowanie prowadzi czytelnika od podstawowych definicji klasy i obiektu, poprzez omówienie czterech fundamentalnych filarów OOP (hermetyzacja, dziedziczenie, polimorfizm, abstrakcja), aż po zaawansowane dobre praktyki, takie jak zasady SOLID oraz różnice między dziedziczeniem a kompozycją. Całość zwieńczona jest przeglądem literatury branżowej oraz propozycjami projektów praktycznych.
\end{abstract}

\tableofcontents

\chapter{Wprowadzenie do Programowania Obiektowego}

\section{Czym jest Programowanie Obiektowe (OOP)?}
Programowanie obiektowe (z ang. \textit{Object-Oriented Programming}, w skrócie OOP) to paradygmat programowania, który zrewolucjonizował sposób tworzenia oprogramowania. Zamiast myśleć o programie jako o sekwencji instrukcji i funkcji operujących na danych, OOP proponuje modelowanie rzeczywistości za pomocą \textbf{obiektów}.

Wyobraźmy sobie programowanie proceduralne jako listę zadań dla kucharza: ,,Weź mąkę'', ,,Dodaj wodę'', ,,Wymieszaj''. Jeśli coś pójdzie nie tak, cały proces może się załamać. Programowanie obiektowe jest jak zorganizowana kuchnia. Mamy obiekty: ,,Kucharz'', ,,Piekarnik'', ,,Miska''. Każdy obiekt ma swoje własne dane (\textbf{atrybuty}) oraz własne funkcje (\textbf{metody}).

\section{Główne Założenia i Korzyści}
Paradygmat obiektowy opiera się na idei łączenia danych oraz funkcji w spójne jednostki. Prowadzi to do wielu korzyści:

\begin{itemize}
    \item \textbf{Modułowość:} Każdy obiekt jest niezależną jednostką.
    \item \textbf{Wielokrotne użycie kodu:} Raz zdefiniowana klasa może być używana wielokrotnie.
    \item \textbf{Łatwiejsze utrzymanie:} Modyfikujesz tylko jedną klasę zamiast przeszukiwać cały kod.
    \item \textbf{Lepsze odwzorowanie rzeczywistości:} Struktura kodu jest bardziej intuicyjna.
    \item \textbf{Elastyczność:} Systemy mogą łatwo adaptować się do nowych typów danych dzięki polimorfizmowi.
\end{itemize}

\section{Klasa vs. Obiekt: Plan i Budynek}
Dwa najbardziej fundamentalne pojęcia w OOP to \textbf{klasa} i \textbf{obiekt}.

\subsection{Klasa (Class)}
To jest plan, szablon lub projekt. Klasa \texttt{Samochod} definiuje, że samochód \textit{będzie miał} kolor i markę oraz że \textit{będzie potrafił} jechać. Sama klasa nie jest samochodem – to tylko opis.

\subsection{Obiekt (Object)}
To jest konkretna, fizyczna \textbf{instancja} klasy. Na podstawie klasy \texttt{Samochod} możemy stworzyć wiele obiektów (np. czerwone Ferrari, niebieski Fiat). Oba obiekty mają te same atrybuty i metody, ale wartości ich atrybutów są niezależne.

Kiedy tworzymy obiekt (proces zwany \textbf{instancjacją}), używamy specjalnej metody zwanej \textbf{konstruktorem}, która pozwala ustawić początkowe wartości atrybutów.
\chapter{Cztery Filary Programowania Obiektowego}

Siła programowania obiektowego opiera się na czterech fundamentalnych koncepcjach.

\section{Hermetyzacja (Enkapsulacja)}
Hermetyzacja to idea łączenia danych i metod w jedną całość oraz \textbf{ukrywanie wewnętrznego stanu obiektu} przed światem zewnętrznym. Inne obiekty nie powinny mieć bezpośredniego dostępu do atrybutów. Dostęp odbywa się przez publiczne metody (gettery i settery).

\textbf{Przykład:} W klasie \texttt{KontoBankowe} atrybut \texttt{saldo} jest prywatny. Metoda ,,wypłać'' sprawdza poprawność operacji przed zmianą stanu, co chroni obiekt przed nieprawidłowym użyciem.

\section{Dziedziczenie}
Dziedziczenie pozwala tworzyć nową klasę (pochodną) na podstawie istniejącej klasy (bazowej). Klasa pochodna dziedziczy atrybuty i metody, mogąc je rozszerzać lub nadpisywać.

\textbf{Przykład:} Klasy \texttt{Samochod} i \texttt{Motocykl} dziedziczą po klasie \texttt{Pojazd}. Modeluje to relację ,,jest'' (ang. \textit{is-a}).

\section{Polimorfizm}
Polimorfizm to zdolność obiektów różnych klas do odpowiadania na to samo wywołanie metody w sposób specyficzny dla ich typu.

\textbf{Przykład:} Metoda ,,jedź'' działa inaczej dla klasy \texttt{Samochod} i inaczej dla klasy \texttt{Motocykl}, ale wywołujemy ją tak samo.

\section{Abstrakcja}
Abstrakcja to proces ukrywania złożonych szczegółów implementacyjnych i pokazywania użytkownikowi tylko niezbędnych funkcji.

\textbf{Przykład:} Używasz pedału gazu (interfejs), nie martwiąc się o wtrysk paliwa (implementacja). W kodzie realizuje się to przez klasy abstrakcyjne lub interfejsy.
\chapter{Zaawansowane Koncepcje i Dobre Praktyki}

Efektywne programowanie wymaga znajomości wzorców i zasad, które pomagają tworzyć kod elastyczny.

\section{Konstruktory i Destruktory}
\begin{itemize}
    \item \textbf{Konstruktor:} Metoda wywoływana automatycznie przy tworzeniu obiektu (inicjalizacja).
    \item \textbf{Destruktor:} Metoda wywoływana przed zniszczeniem obiektu (zwalnianie zasobów).
\end{itemize}

\section{Dziedziczenie a Kompozycja}
Często lepszym podejściem od dziedziczenia (relacja ,,jest'') jest \textbf{kompozycja} (relacja ,,ma'').
\begin{itemize}
    \item Dziedziczenie: \texttt{Samochod} \textbf{jest} \texttt{Pojazdem}.
    \item Kompozycja: \texttt{Samochod} \textbf{ma} \texttt{Silnik}.
\end{itemize}
Kompozycja jest bardziej elastyczna i pozwala na luźniejsze powiązania między klasami.

\subsection{Porównanie: Dziedziczenie vs Kompozycja}
Poniższa tabela podsumowuje kluczowe różnice:

\begin{table}[h]
\centering
\renewcommand{\arraystretch}{1.5}
\begin{tabularx}{\textwidth}{|l|X|X|}
\hline
\textbf{Cecha} & \textbf{Dziedziczenie (,,IS-A'')} & \textbf{Kompozycja (,,HAS-A'')} \\ \hline
\textbf{Definicja} & Klasa pochodna dziedziczy po bazowej. & Klasa zawiera instancję innej klasy. \\ \hline
\textbf{Wiązanie} & Silne (\textit{tight coupling}). & Luźne (\textit{loose coupling}). \\ \hline
\textbf{Elastyczność} & Mniejsza (ustalana przy kompilacji). & Większa (możliwa wymiana w trakcie działania). \\ \hline
\end{tabularx}
\caption{Porównanie dziedziczenia i kompozycji.}
\end{table}

\section{Zasady SOLID}
Zbiór pięciu zasad SOLID pomaga tworzyć zrozumiałe oprogramowanie:
\begin{enumerate}
    \item \textbf{S (Single Responsibility):} Jedna klasa, jedna odpowiedzialność.
    \item \textbf{O (Open/Closed):} Otwarte na rozszerzenia, zamknięte na modyfikacje.
    \item \textbf{L (Liskov Substitution):} Podklasy muszą móc zastąpić klasy bazowe.
    \item \textbf{I (Interface Segregation):} Wiele małych interfejsów zamiast jednego dużego.
    \item \textbf{D (Dependency Inversion):} Zależność od abstrakcji, nie od konkretów.
\end{enumerate}
\chapter{Co Dalej? Nauka w Praktyce, Projekty i Zasoby}

Zrozumienie teorii to pierwszy krok. Drugim jest praktyka.

\section{Przykładowe Projekty do Nauki OOPs}
\begin{itemize}
    \item \textbf{Projekt 1: Prosty System Biblioteczny.} Ćwiczy podstawy klas i hermetyzację.
    \item \textbf{Projekt 2: Gra Tekstowa (RPG).} Zastosowanie dziedziczenia i polimorfizmu.
    \item \textbf{Projekt 3: Symulator Bankomatu.} Ćwiczy kompozycję i obsługę wyjątków.
\end{itemize}

\section{Klasyczne Książki o OOPs}
Warto sięgnąć po literaturę uznawaną za kanon:

\begin{itemize}
    \item \textbf{,,Wzorce Projektowe''} (Banda Czterech) \cite{gof}.
    \item \textbf{,,Czysty kod''} (Robert C. Martin) \cite{cleancode}.
    \begin{figure}[h]
    \centering
    \includegraphics[width=0.4\textwidth]{image1.png}
    \caption{Okładka książki ,,Czysty kod'' Roberta C. Martina.}
    \label{fig:cleancode}
\end{figure}
    \item \textbf{,,Rusz głową! Wzorce projektowe''} \cite{headfirst}.
    \item \textbf{,,Refaktoryzacja''} (Martin Fowler) \cite{refactoring}.
\end{itemize}

\section{Inne Zasoby}
Warto również korzystać z platform e-learningowych, oficjalnej dokumentacji oraz analizować kod open source na GitHubie.
\chapter{wnioski}

W tym projekcie, podczas doboru klasy do tekstu w latexie, podjęłem decyzję o wyborze klasy dokumentu \texttt{report}, rezygnując z klasy \texttt{book}.

Główne powody, dla których klasa \texttt{report} jest lepszym rozwiązaniem w tym przypadku:

\begin{enumerate}
    \item \textbf{Struktura dokumentu:} Klasa \texttt{book} jest przeznaczona do składu pełnowymiarowych książek. Dzieli ona treść na części (\textit{parts}), rozdziały (\textit{chapters}) i sekcje, a także wprowadza podział na \textit{frontmatter} (wstęp), \textit{mainmatter} (treść główna) i \textit{backmatter} (dodatki). Dla pracy o objętości kilkunastu stron taka struktura jest nadmiarowa. Klasa \texttt{report} idealnie sprawdza się w przypadku prac dyplomowych i raportów technicznych, które składają się z kilku rozdziałów, ale nie wymagają skomplikowanej struktury książkowej.
    
    \item \textbf{Formatowanie stron (Jednostronne vs Dwustronne):} Domyślnie klasa \texttt{book} zakłada druk dwustronny. Oznacza to, że marginesy na stronach parzystych i nieparzystych są różne (szerszy margines wewnętrzny na oprawę), a każdy nowy rozdział musi zaczynać się na stronie nieparzystej (prawej). Powoduje to powstawanie wielu pustych stron w krótkich dokumentach. Klasa \texttt{report} jest bardziej elastyczna i czytelna przy przeglądaniu dokumentu na ekranie komputera, nie generując zbędnych pustych stron.
    
    \item \textbf{Nagłówki i stopki:} W klasie \texttt{book} domyślne nagłówki są przygotowane pod druk dwustronny (tytuł rozdziału na lewej stronie, tytuł sekcji na prawej). W krótszej pracy prostszy układ nagłówków oferowany przez klasę \texttt{report} zapewnia większą przejrzystość i jest łatwiejszy w konfiguracji.
\end{enumerate}
\section*{Dostęp do kodu źródłowego}
Pełny kod źródłowy projektu oraz pliki niniejszej dokumentacji są dostępne w repozytorium GitHub pod adresem:

\vspace{0.5cm} % Mały odstęp
\noindent
\url{https://github.com/twoj-login/twoje-repozytorium}



\begin{thebibliography}{9}

\bibitem{gof}
Gamma E., Helm R., Johnson R., Vlissides J.,
\textit{Wzorce projektowe. Elementy oprogramowania obiektowego wielokrotnego użytku}.
Helion.

\bibitem{cleancode}
Martin R. C.,
\textit{Czysty kod. Podręcznik dobrego programisty}.
Helion.

\bibitem{headfirst}
Freeman E., Robson E.,
\textit{Rusz głową! Wzorce projektowe}.
Helion.

\bibitem{refactoring}
Fowler M.,
\textit{Refaktoryzacja. Ulepszanie struktury istniejącego kodu}.
Helion.

\end{thebibliography}

\end{document}